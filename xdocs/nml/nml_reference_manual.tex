\documentclass[oneside,final,14pt]{extreport}
\usepackage[utf8]{inputenc}
\usepackage[T2A]{fontenc}
\usepackage[english]{babel}
\usepackage{vmargin}
\usepackage{listings}
\setpapersize{A4}
\usepackage{indentfirst}


\lstset{
 language=C,
 frame=single,
 basicstyle=\ttfamily\scriptsize,
 commentstyle=\itshape,
 emph={op, mode, reg, alias, mem, type, let, card, int, syntax, image, format, action,
       if, then, else, elif, endif, coerce, exception, address, segment, buffer, register,
       memory, mmu, range, ways, sets, match, tag, index, entry, policy, read, write,
       var, hit, path, transition, guard},
 emphstyle={\bfseries}}


\begin{document}

\begin{titlepage}
\begin{center}
\Large{Institute for System Programming of the Russian Academy of Sciences}

\vfill


\bf\Large{nML Reference Manual}

(UNDER DEVELOPMENT)

\vfill

\bf
Moscow 2016
\end{center}


\end{titlepage}

\newpage
\stepcounter{page} % Increase page counter by one
\tableofcontents

\newpage


\chapter{Chapter}

\section{Introduction}

\section{Constants}

\section{Data Types}

A data type specifies the format of values stored in registers or memory. nML supports
the following data types: 

\begin{itemize}

\item int(N): N-bit signed integer data type. Negative numbers are stored in two's 
complement form. The range of possible values is [-2n-1 ... 2n-1 - 1].

\item card(N): N-bit unsigned integer data type. The range of possible values is [0 ... 2n - 1].

\item float(N, M): IEEE 754 floating point number, where fraction size is N and exponent size is M.
The resulting type size is N + M + 1 bits, where 1 is an implicitly added bit for store the
sign. Supported floating-point formats include:

\begin{itemize}
\item 32-bit single-precision. Defined as float(23, 8).
\item 64-bit double-precision. Defined as float(52, 11).
\item 80-bit double-extended-precision. Defined as float(64, 15).
\item 128-bit quadruple-precision. Defined as float(112, 15).
\end{itemize}

\end{itemize}

nML allows declaring aliases for data types. Here is a simple type declaration:

\begin{lstlisting}
type DWORD = card(32)
\end{lstlisting}

In this example, type is a reserved word, DWORD is the declared alias type name, the card(32) is
the actual data type.

\lstinputlisting{examples/test.nml}

\addcontentsline{toc}{section}{\numberline{}Introduction}

\end{document}
