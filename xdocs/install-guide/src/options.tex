\chapter{Options}

%%%%%%%%%%%%%%%%%%%%%%%%%%%%%%%%%%%%%%%%%%%%%%%%%%%%%%%%%%%%%%%%%%%%%%%%%%%%%%%%%%%%%%%%%%%%%%%%%%%
\section{Main Command-Line Options}
\label{Command_Line_Options}

MicroTESK works in two modes: \emph{specification translation} and \emph{test generation},
which are enabled with the \texttt{--translate} (used by default) and \texttt{--generate} keys
correspondingly. In addition, the \texttt{--help} key prints information on the command-line format.

The \texttt{--translate} and \texttt{--generate} keys are inserted into the command-line by
\texttt{compile.sh}/\texttt{compile.bat} and \texttt{generate.sh}/\texttt{generate.bat} scripts
correspondingly.

%%%%%%%%%%%%%%%%%%%%%%%%%%%%%%%%%%%%%%%%%%%%%%%%%%%%%%%%%%%%%%%%%%%%%%%%%%%%%%%%%%%%%%%%%%%%%%%%%%%
\section{Other Command-Line Options}

Other options should be specified explicitly to customize the behavior of MicroTESK.
Here is the list of options:\\

\begin{tabular}{ | p{4cm} | p{1cm} | p{5cm} | p{2.5cm} |}
  \hline
  \textbf{Full name} & \textbf{Short name} & \textbf{Description} & \textbf{Requires} \\ \hline
  --help & -h & Shows help message & \\ \hline
  --verbose & -v & Enables printing diagnostic messages & \\ \hline
  --translate & -t & Translates formal specifications & \\ \hline
  --generate & -g & Generates test programs & \\ \hline
  --output-dir <arg> & -od & Sets where to place generated files & \\ \hline
  --include <arg> & -i & Sets include files directories & --translate \\ \hline
  --extension-dir <arg> & -ed & Sets directory that stores user-defined Java code & --translate \\ \hline
  --random-seed <arg> & -rs & Sets seed for randomizer & --generate \\ \hline
  --solver <arg> & -s & Sets constraint solver engine to be used & --generate \\ \hline
  --branch-exec-limit <arg> & -bel & Sets the limit on control transfers to detect endless loops & --generate \\ \hline
  --solver-debug & -sd & Enables debug mode for SMT solvers & --generate \\ \hline
  --tarmac-log & -tl & Saves simulator log in Tarmac format & --generate \\ \hline
  --self-checks & -sc & Inserts self-checking code into test programs & --generate \\ \hline
  --default-test-data & -dtd & Enables generation of default test data & --generate \\ \hline
  --arch-dirs <arg> & -ad & Home directories for tested architectures & --generate \\ \hline
  --rate-limit <arg> & -rl & Generation rate limit, causes error when broken & --generate \\ \hline
  --code-file-extension <arg> & -cfe & The output file extension & --generate \\ \hline
  --code-file-prefix <arg> & -cfp & The output file prefix (file names are as follows prefix{\_}xxxx.ext, where xxxx is a 4-digit decimal number) & --generate \\ \hline
  --data-file-extension <arg> & -dfe & The data file extension & --generate \\ \hline
  --data-file-prefix <arg> & -dfp & The data file prefix & --generate \\ \hline
\end{tabular}

\begin{tabular}{ | p{4cm} | p{1cm} | p{5cm} | p{2.5cm} |}
  \hline
  --exception-file-prefix <arg> & -efp & The exception handler file prefix & --generate \\ \hline
  --program-length-limit <arg> & -pll & The maximum number of instructions in output programs & --generate \\ \hline
  --trace-length-limit <arg> & -tll & The maximum length of execution traces of output programs & --generate \\ \hline
  --comments-enabled & -ce & Enables printing comments; if not specified no comments are printed & --generate \\ \hline
  --comments-debug & -cd & Enables printing detailed comments; must be used together with --comments-enabled  & --generate \\ \hline
  --no-simulation & -ns & Disables simulation of generated test programs on the model & --generate \\ \hline
  --time-statistics & -ts & Enables printing time statistics & --generate \\ \hline
\end{tabular}

%%%%%%%%%%%%%%%%%%%%%%%%%%%%%%%%%%%%%%%%%%%%%%%%%%%%%%%%%%%%%%%%%%%%%%%%%%%%%%%%%%%%%%%%%%%%%%%%%%%
\section{Settings File}

Default values of options are stored in the \texttt{<MICROTESK{\_}HOME>/etc/settings.xml}
configururation file that has the following format:

\begin{lstlisting}[language=ruby]
<?xml version="1.0" encoding="utf-8"?>
<settings>
  <setting name="random-seed" value="0"/>
  <setting name="branch-exec-limit" value="1000"/>
  <setting name="code-file-extension" value="asm"/>
  <setting name="code-file-prefix" value="test"/>
  <setting name="data-file-extension" value="dat"/>
  <setting name="data-file-prefix" value="test"/>
  <setting name="exception-file-prefix" value="test_except"/>
  <setting name="program-length-limit" value="1000"/>
  <setting name="trace-length-limit" value="1000"/>
  <setting name="comments-enabled" value=""/>
  <setting name="comments-debug" value=""/>
  <setting name="default-test-data" value=""/>
  <setting
    name="arch-dirs" 
    value="cpu=arch/demo/cpu/settings.xml:minimips=arch/minimips/settings.xml"
  />
</settings>
\end{lstlisting}
