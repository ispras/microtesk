\documentclass[oneside,final,14pt]{extreport}

\usepackage[utf8]{inputenc}
\usepackage[T2A]{fontenc}
\usepackage[english]{babel}
\usepackage{vmargin}
\usepackage{listings}
\setpapersize{A4}
\usepackage{indentfirst}
\usepackage{graphicx}

\lstset{
 language=C,
 frame=single,
 basicstyle=\ttfamily\scriptsize,
 commentstyle=\itshape,
 emph={op, mode, reg, alias, mem, type, let, card, int, syntax, image, format, action,
       if, then, else, elif, endif, coerce, exception, address, segment, buffer, register,
       memory, mmu, range, ways, sets, match, tag, index, entry, policy, read, write,
       var, hit, path, transition, guard},
 emphstyle={\bfseries}}

%%%%%%%%%%%%%%%%%%%%%%%%%%%%%%%%%%%%%%%%%%%%%%%%%%%%%%%%%%%%%%%%%%%%%%%%%%%%%%%%%%%%%%%%%%%%%%%%%%%

\begin{document}

\begin{titlepage}
\begin{center}
\Large{Institute for System Programming of the Russian Academy of Sciences}

\vfill


\bf\Large{MicroTESK User Guide}

(UNDER DEVELOPMENT)

\vfill

\bf
Moscow 2016
\end{center}
\end{titlepage}

%%%%%%%%%%%%%%%%%%%%%%%%%%%%%%%%%%%%%%%%%%%%%%%%%%%%%%%%%%%%%%%%%%%%%%%%%%%%%%%%%%%%%%%%%%%%%%%%%%%

\newpage
\stepcounter{page} % Increase page counter by one
\tableofcontents

%%%%%%%%%%%%%%%%%%%%%%%%%%%%%%%%%%%%%%%%%%%%%%%%%%%%%%%%%%%%%%%%%%%%%%%%%%%%%%%%%%%%%%%%%%%%%%%%%%%

\chapter{Installation}

%%%%%%%%%%%%%%%%%%%%%%%%%%%%%%%%%%%%%%%%%%%%%%%%%%%%%%%%%%%%%%%%%%%%%%%%%%%%%%%%%%%%%%%%%%%%%%%%%%%
\section{System Requirements}

MicroTESK is a set of Java-based utilities that are run from the command line.
It can be used on \textbf{\textit{Windows}}, \textbf{\textit{Linux}} and
\textbf{\textit{OS X}} machines that have \textbf{\textit{JDK 1.7 or later}}
installed. To build MicroTESK from source code or to build the generated
Java models, \textbf{\textit{Apache Ant version 1.8 or later}} is required.
To generate test data based on constraints, MicroTESK needs
the \textbf{\textit{Microsoft Research Z3}} or \textbf{\textit{CVC4}} solver that
can work under the corresponding operating system.

%%%%%%%%%%%%%%%%%%%%%%%%%%%%%%%%%%%%%%%%%%%%%%%%%%%%%%%%%%%%%%%%%%%%%%%%%%%%%%%%%%%%%%%%%%%%%%%%%%%
\section{Running MicroTESK}

To generate a Java model of a microprocessor from its nML specification, a user
needs to run the compile.sh script (Unix, Linux, OS X) or the compile.bat script
(Windows). For example, the following command generates a model for the miniMIPS
specification:

\begin{lstlisting}[language=bash]
sh bin/compile.sh arch/minimips/model/minimips.nml
\end{lstlisting}

NOTE: Models for all demo specifications are already built and included in the
MicroTESK distribution package. So a user can start working with MicroTESK from
generating test programs for these models.

To generate a test program, a user needs to use the generate.sh script
(Unix, Linux, OS X) or the generate.bat script (Windows). The scripts
require the following parameters:

\begin{enumerate}
  \item model name
  \item test template file
  \item target test program source code file
\end{enumerate}

For example, the command below runs the euclid.rb test template for
the miniMIPS model generated by the command from the previous example and saves
the generated test program to an assembler file. The file name is based on values
of the --code-file-prefix and --code-file-extension options.

\begin{lstlisting}[language=bash]
sh bin/generate.sh minimips arch/minimips/templates/euclid.rb
\end{lstlisting}

To specify whether Z3 or CVC4 should be used to solve constraints,
a user needs to specify the -s or --solver command-line option as z3
or cvc4 respectively. By default, Z3 will be used. Here is an example:

\begin{lstlisting}[language=bash]
sh bin/generate.sh -s cvc4 minimips arch/minimips/templates/constraint.rb
\end{lstlisting}

More information on command-line options can be found on the Command-Line Options
section.

%%%%%%%%%%%%%%%%%%%%%%%%%%%%%%%%%%%%%%%%%%%%%%%%%%%%%%%%%%%%%%%%%%%%%%%%%%%%%%%%%%%%%%%%%%%%%%%%%%%
\section{Command-Line Options}

MicroTESK works in two modes: specification translation and test generation,
which are enabled with the --translate (used by default) and --generate keys
correspondingly. In addition, the --help key prints information on the command-line format.

The --translate and --generate keys are inserted into the command-line by
compile.sh/compile.bat and generate.sh/generate.bat scripts correspondingly.
Other options should be specified explicitly to customize the behavior of MicroTESK.
Here is the list of options:

\begin{tabular}{ | p{4cm} | p{1cm} | p{5cm} | p{3cm} |}
  \hline
  Full name & Short name & Description & Requires \\ \hline
  --help & -h & Shows help message & \\ \hline
  --verbose & -v & Enables printing diagnostic messages & \\ \hline
  --translate & -t & Translates formal specifications & \\ \hline
  --generate & -g & Generates test programs & \\ \hline
  --output-dir <arg> & -od & Sets where to place generated files & \\ \hline
  --include <arg> & -i & Sets include files directories & --translate \\ \hline
  --extension-dir <arg> & -ed & Sets directory that stores user-defined Java code & --translate \\ \hline
  --random-seed <arg> & -rs & Sets seed for randomizer & --generate \\ \hline
  --solver <arg> & -s & Sets constraint solver engine to be used & --generate \\ \hline
  --branch-exec-limit <arg> & -bel & Sets the limit on control transfers to detect endless loops & --generate \\ \hline
  --solver-debug & -sd & Enables debug mode for SMT solvers & --generate \\ \hline
  --tarmac-log  & -tl & Saves simulator log in Tarmac format & --generate \\ \hline
  --self-checks & -sc & Inserts self-checking code into test programs & --generate \\ \hline
  --arch-dirs <arg> & -ad & Home directories for tested architectures & --generate \\ \hline
  --rate-limit <arg> & -rl & Generation rate limit, causes error when broken & --generate \\ \hline
  --code-file-extension <arg> & -cfe & The output file extension & --generate \\ \hline
  --code-file-prefix <arg> & -cfp & The output file prefix (file names are as follows prefix{\_}xxxx.ext, where xxxx is a 4-digit decimal number) & --generate \\ \hline
  --data-file-extension <arg> & -dfe & The data file extension & --generate \\ \hline
  --data-file-prefix <arg> & -dfp & The data file prefix & --generate \\ \hline
  --exception-file-prefix <arg> & -efp & The exception handler file prefix & --generate \\ \hline
  --program-length-limit <arg> & -pll & The maximum number of instructions in output programs & --generate \\ \hline
  --trace-length-limit <arg> make& -tll & The maximum length of execution traces of output programs & --generate \\ \hline
  --comments-enabled & -ce & Enables printing comments; if not specified no comments are printed & --generate \\ \hline
  --comments-debug & -cd & Enables printing detailed comments; must be used together with --comments-enabled  & --generate \\ \hline
\end{tabular}

%%%%%%%%%%%%%%%%%%%%%%%%%%%%%%%%%%%%%%%%%%%%%%%%%%%%%%%%%%%%%%%%%%%%%%%%%%%%%%%%%%%%%%%%%%%%%%%%%%%

\section{Overview}

%%%%%%%%%%%%%%%%%%%%%%%%%%%%%%%%%%%%%%%%%%%%%%%%%%%%%%%%%%%%%%%%%%%%%%%%%%%%%%%%%%%%%%%%%%%%%%%%%%%

\chapter{Appendixes}

\section{References}

% Related nML publications/documents

%%%%%%%%%%%%%%%%%%%%%%%%%%%%%%%%%%%%%%%%%%%%%%%%%%%%%%%%%%%%%%%%%%%%%%%%%%%%%%%

\addcontentsline{toc}{section}{\numberline{}Bibliography}
\begin{thebibliography}{1}

\bibitem{freericks}
M.~Freericks.
\emph{The nML Machine Description Formalism}.
Technical Report TR SM-IMP/DIST/08, TU Berlin CS Department, 1993.

\end{thebibliography}

\end{document}
