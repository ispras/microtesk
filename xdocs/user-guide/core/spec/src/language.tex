\chapter{Language Facilities}

%%%%%%%%%%%%%%%%%%%%%%%%%%%%%%%%%%%%%%%%%%%%%%%%%%%%%%%%%%%%%%%%%%%%%%%%%%%%%%%%%%%%%%%%%%%%%%%%%%%

\section{Constants}

In nML, constants are declared using the \textbf{\textit{let}} keyword. For example,
the following line of code declares a constant called {\texttt A} that has the value {\texttt 100}:

\begin{lstlisting}
let A = 100
\end{lstlisting}

Constants are global. They can be defined only once and can be used in any context
their values could stand.

Constants can help extend nML. Any information about a machine that can be given with
a single number or string can easily be defined as a constant.

\subsection{Constant Data Types}

Constants can be represented by integer numbers of inifinite precision and strings.
Integer constants are cast to specific nML data types depending on the context in
which they are used.

%%%%%%%%%%%%%%%%%%%%%%%%%%%%%%%%%%%%%%%%%%%%%%%%%%%%%%%%%%%%%%%%%%%%%%%%%%%%%%%%%%%%%%%%%%%%%%%%%%%

\section{Data Types}

A data type specifies the format of values stored in registers or memory.
nML supports the following data types:

\begin{itemize}

\item \textbf{\textit{int(N)}}:
N-bit signed integer data type. Negative numbers are stored in two's complement
form. The range of possible values is [-2$^{n-1}$ ... 2$^{n-1}$ - 1].

\item \textbf{\textit{card(N)}}:
N-bit unsigned integer data type. The range of possible values is [0 ... 2$^n$ - 1].

\item \textbf{\textit{float(N, M)}}: IEEE 754 floating point number, where fraction
size is N and exponent size is M. The resulting type size is N + M +1 bits, where 1
is an implicitly added bit for store the sign. Supported floating-point formats include:

\begin{itemize}
\item 32-bit single-precision. Defined as float(23, 8).
\item 64-bit double-precision. Defined as float(52, 11).
\item 80-bit double-extended-precision. Defined as float(64, 15).
\item 128-bit quadruple-precision. Defined as float(112, 15).
\end{itemize}

\end{itemize}

nML allows declaring aliases for data types using the \textbf{\textit{type}} keyword.
For example, in the code below, WORD is declared as an alias for card(32):

\begin{lstlisting}
type WORD = card(32)
\end{lstlisting}

\subsection{Converting Data Types}

In nML, type conversion is performed explicitly using special functions.
Implicit type conversion is not supported. Type conversion functions:

\begin{itemize}
\item \textbf{\textit{sign{\_}extend(type, value)}}:
Sign-extends the value to the specified type. Applied to any types. Requires
the new type to be larger or equal to the original type.

\item \textbf{\textit{zero{\_}extend(type, value)}}:
Zero-extends the value to the specified type. Applied to any types. Requires
the new type to be larger or equal to the original type.

\item \textbf{\textit{coerce(type, value)}}:
Converts an integer value to another integer type. For smaller types the value
is truncated, for larger types it is extended. If the original type is a signed
integer, sign extension is performed. Otherwise, zero-extension is performed.

\item \textbf{\textit{cast(type, value)}}:
Reinterprets the value as the specified type. Applied to any types. The new type
must be of the same size as the original type.

\item \textbf{\textit{int{\_}to{\_}float(type, value)}}:
Converts 32 and 64-bit integers into 32, 64, 80 and 128-bit floating-point
values (IEEE 754).

\item \textbf{\textit{float{\_}to{\_}int(type, value)}}:
Converts 32, 64, 80 and 128-bit floating-point values (IEEE 754) into 32 and
64-bit integers.

\item \textbf{\textit{float{\_}to{\_}float(type, value)}}:
Converts 32, 64, 80 and 128-bit floating-point values (IEEE 754) into each
other.

\end{itemize}

%%%%%%%%%%%%%%%%%%%%%%%%%%%%%%%%%%%%%%%%%%%%%%%%%%%%%%%%%%%%%%%%%%%%%%%%%%%%%%%%%%%%%%%%%%%%%%%%%%%

\section{Memory, Registers and Variables}

Memory and registers represent the visible state of a microprocessor. Only this state is carried
across execution of instruction calls. Both entities are modeled as arrays of the specified
data type. Although memory is external to the microprocessor while registers are internal, they
are described in nML in the same way. Differences are related to intergration with the MMU model
and are transparent to users.
In addition to memory and registers, nML supports defining temporary variables, which can be used
to exchange data between primitives that constitute a microprocessor instruction. Such variables
do not represent any externally visible state and do not save their values across instruction
calls.

\subsection{Memory}

Memory is described as a data array that represents the state of physical memory. Virtual memory
issues such as address translation, caching, etc. are not covered by nML specifications.

A memory array is defined using the \textbf{\textit{mem}} keyword. For example, the code below
defines a memory array called {\texttt MEM} that consists of 2$^{30}$ words:

\begin{lstlisting}
mem MEM [2 ** 30, WORD]
\end{lstlisting}

The array lenght must be specified as a contant expression. In the initial state, the array is
filled with zeros.

\subsection{Registers}

Registers are defined using the \textbf{\textit{reg}} keyword. Like memory, they are described
as an array of the specified type. For example, the following code defines a register array that
includes 32 registers storing a word:

\begin{lstlisting}
mem GPR[32, WORD]
\end{lstlisting}

The array length must be specified as a constant expression. Uninitialized registers hold
zeros. When the array contains only a single element, the length expression can be skipped:

\begin{lstlisting}
reg HI[WORD]
\end{lstlisting}

\subsection{Variables}

Variables are defined using the \textbf{\textit{var}} keyword. Their role is to store
temporary values used by instructions. They are similar to registers, but they do not represent
any visible state and do not save their values across instruction calls. Variables are initialized
with zeros and reset their values after the execution of an instruction call. For example, the code
below defines a variable for storing a 33-bit signed integer value:

\begin{lstlisting}
var temp[int(33)]
\end{lstlisting}


\subsection{Aliases}


%%%%%%%%%%%%%%%%%%%%%%%%%%%%%%%%%%%%%%%%%%%%%%%%%%%%%%%%%%%%%%%%%%%%%%%%%%%%%%%%%%%%%%%%%%%%%%%%%%%

\section{Operators}

Here is the list of predefined nML operators:

\begin{itemize}

\item \textbf{Binary +, --}

\item \textbf{Unary +, --}

\item \textbf{$\ast$, /, \%}

\item \textbf{$\ast\ast$}

\item \textbf{<, >, <=, >=, ==, !=}

\item \textbf{{<}{<}, {>}{>}, \&, |, \^{}, \~{}}

\item \textbf{{<}{<}{<}, {>}{>}{>}}

\item \textbf{!, \&\&, ||}

\item \textbf{::}

\item \textbf{sqrt}

\item \textbf{is{\_}nan}

\item \textbf{is{\_}signaling{\_}nan}

\end{itemize}

%%%%%%%%%%%%%%%%%%%%%%%%%%%%%%%%%%%%%%%%%%%%%%%%%%%%%%%%%%%%%%%%%%%%%%%%%%%%%%%%%%%%%%%%%%%%%%%%%%%

\section{Special Functions}

TODO:\\

\textbf{format(format-string, args*)}:
Used to put together the string values of the \textbf{\textit{syntax}} and \textbf{\textit{image}}
attributes. The format string is a variation of the printf format string well known from C.
It may contain alphanumeric characters, blanks, tabs ('\textbackslash t'), newlines
('\textbackslash n'), and format directives of the form \%nC, where n is an optional field size
and C is one of the following characters:

\begin{itemize}

\item  d This takes an Integer or Cardinal argument from the argument list and formats it as a
         decimal number.

\item  x This takes a Cardinal argument and formats it as a hexadecimal number.

\item  b This takes a Cardinal argument and formats it as a binary number. It may also take a
         binary string, i.e. a string containing only 1s, 0s and (ignored) whitespace.

\item  s This takes a string argument and incorporates it as a whole.

\end{itemize}

