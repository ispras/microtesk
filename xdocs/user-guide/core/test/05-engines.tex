\chapter{Test Engines}

\section{Branch Engine}

\subsection{Parameters}

\begin{itemize}
\item \emph{branch{\_}exec{\_}limit} is an upper bound for the number of executions of a single branch instruction;
\item \emph{trace{\_}count{\_}limit} is an upper bound for the number of execution traces to be returned.
\end{itemize}

More information on the parameters is given in the “Execution Traces Enumeration” section.

\subsection{Description}

Functioning of the \emph{branch} test engine includes the following steps:

\begin{enumerate}
  \item construction of a \emph{branch structure} of an abstract test sequence;
  \item enumeration of \emph{execution traces} of the branch structure;
  \item concretization of the test sequence for each execution trace:
  \begin{enumerate}
    \item construction of a \emph{control} code;
    \item construction of an \emph{initialization} code.
  \end{enumerate}
\end{enumerate}

Let \emph{D} be the size of the delay slot for an architecture under scrutiny (e.g., \emph{D}=1
for MIPS, and \emph{D}=0 for ARM).
