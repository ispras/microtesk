\chapter{Simulator Memory Configuration}

NOTE: Features described in the present document are subject to review.
Mechanisms of address allocation for code and data sections initially were developed as separate features.
Data were stored in physical memory, while instructions were stored in a separate internal tables using virtual addresses and were not actually stored and fetched from physical memory.
Now these mechanisms are in the process of unification. For this reason, at the current stage, the scheme is a bit awkward.

Addresses used by MicroTESK to allocate code and data sections are configured using the following settings:

\begin{itemize}
\item
\texttt{base\_virtual\_address} (by default, equals 0);
\item
\texttt{base\_physical\_address} (by default, equals 0).
\end{itemize}

The settings are initialized in the initialize method of a template in the following way:

\begin{lstlisting}
def initialize
  super
  # Memory-related settings
  set_option_value 'base-virtual-address', 0x00001000
  set_option_value 'base-physical-address', 0x00001000
end
\end{lstlisting}

In addition, the configuration of physical memory and data allocation constructs is set up using the \texttt{data\_config} construct.
For example:

\begin{lstlisting}
data_config(:text => '.data',
            :target => 'M',
            :base_virtual_address => 0x40180000) {
  ...
}
\end{lstlisting}

The construct specifies memory array used as physical memory (\texttt{:target}) and base virtual address used for data allocation (\texttt{:base\_virtual\_address}).

Schemes of address allocation for code and data sections are described in detail further in this document.

\section{Code Sections}

Memory modeling for code sections is optional.
It is enabled using the \texttt{--fetch-decode-enabled} (\texttt{-fde}) option.
If it is not enabled, instructions are not stored in physical memory and have no physical addresses.
They are stored in special tables and accessed using their virtual addresses.

\subsection{Disabled memory modeling}

When memory modeling is disabled the scheme is the following.
MicroTESK starts address allocations for code sections at \texttt{base\_virtual\_address}.
The VA for the current allocation is calculated as VA for the previous code allocation + size of previous code allocation.

Allocation address can be modified using the org directive. MicroTESK allows specifying relative and absolute origins:

\begin{itemize}
\item
Relative origin:
\texttt{org :delta => n}, $VA = VA + n$;
\item
Absolute origin:
\texttt{org n}, $VA = base\_virtual\_address + n$.
\end{itemize}

Addresses can be aligned using the align directive. By default, align $n$ means align at the border of $2 \times n$ bytes.
When code is simulated, no fetches are performed.
Instead, MicroTESK extracts instructions stored in special tables using their virtual addresses.

\subsection{Enabled memory modeling}

When memory modeling is enabled, MicroTESK stores binary representation of instructions in physical memory.
Also, objects describing instructions are stored in special tables in the same way as when memory modeling is disabled.
The only difference is that now they are indexed using physical addresses.

Physical addresses used for allocation are calculated using the following formula:

$$PA = base\_physical\_address + origin,$$
where $origin = VA - base\_virtual\_address$

When code is simulated, instruction fetches are performed.
Logic of fetches is described in MMU specifications (address translation and accesses to cache buffers).
For each fetch, MicroTESK gets binary representation of the instruction and its physical address (written to the \texttt{MMU\_PA} register).
First, it tries to extract the instruction from the table using its physical address.
If the instruction is not found (execution of self-modifying code), it tries to decode its binary representation.
If decoding fails, the \texttt{invalid\_instruction} operation defined in the nML specification is called (which throws a corresponding exception).

\section{Data Sections}

There two use cases of addressing data which are handled using separate mechanisms: 

\begin{itemize}
\item
Data is allocated in memory using the \texttt{data{...}} construct.
\item
Data is read or written to memory using load/store instructions. 
\end{itemize}

Loads and stores use the MMU model included into the MicroTESK simulator, which involves address translation and accesses to cache buffers according to the MMU specifications.

For data allocation, MMU logic is not used.

Data are placed directly to physical memory starting from \texttt{base\_physical\_address}.
Address allocation is controlled by the org and align directives.
They work similarly to the one used for code sections, but operate with physical addresses:

\begin{itemize}
\item
Relative origin: \texttt{org :delta => n}, $PA = PA + n$;
\item
Absolute origin: \texttt{org n}, $PA = base\_physical\_address + n$. 
\end{itemize}

Labels in data sections are assigned virtual addresses.
To do this, PA is translated into VA.
Address translation does not use the MMU model.
Instead, it uses a simplified scheme based on settings that works as follows:

$$VA = base\_virtual\_address + (PA - base\_physical\_address).$$

NOTE: Allocation addresses VA and PA are tracked separately for data and code sections.
Therefore, it is required to take care to avoid address conflicts.

MicroTESK allows using different base addresses for code and data allocations.
Data allocations can be assigned a separate base virtual address.
This can be done in the following way (see the pre method of the base test template):

\begin{lstlisting}
data_config(:text => '.data',
            :target => 'M',
            :base_virtual_address => 0x00002000) {
  ...
}
\end{lstlisting}

This VA is translated into PA using the simplified address translation scheme ($PA = base\_physical\_address + (VA - base\_virtual\_address)$),
and the result is used as base PA for data allocations.

Example: \texttt{min\_max.rb}.

NOTE: This is a simplified example.
It considers PA to be equal VA.
This assumption is made because the address translation mechanism is currently disabled in MMU specifications.
Anyway, this should be enough to illustrate the above-described principles.

\subsection{Settings}

Setting values are as follows:

\begin{itemize}
\item
\texttt{base\_virtual\_address} and \texttt{base\_physical\_address} use the default value 0 (are not initialized).
\item
\texttt {base\_virtual\_address} for data sections is specified as $0x40180000$ (see code below).
\end{itemize}

\begin{lstlisting}
data_config(:text => '.data',
            :target => 'M',
            :base_virtual_address => 0x40180000) {
  ...
}
\end{lstlisting}

\subsection{Prologue}

Starting VA for the test program�s main code is specified as $0x2000$ (calculated as $base\_virtual\_address + origin = 0 + 0x2000$):

\begin{lstlisting}
text '.text'
text '.globl _start'
movz x0, vbar_el1_value, 0
msr vbar_el1, x0
bl :_start
org 0x2000
label :_start
\end{lstlisting}

The code is allocated in the simulator�s physical memory in the following way (from MicroTESK debug output):

\begin{lstlisting}
0x00000000 (PA): nop (0xD503201F)
0x00000004 (PA): movz x0, #0x50, LSL #0 (0xD2800A00)
0x00000008 (PA): msr vbar_el1, x0 (0xD518C000)
0x0000000c (PA): movz x0, #0x700, LSL #0 (0xD280E000)
0x00000010 (PA): msr VBAR_EL2, x0 (0xD51CC000)
0x00000014 (PA): movz x0, #0xd50, LSL #0 (0xD281AA00)
0x00000018 (PA): msr VBAR_EL3, x0 (0xD51EC000)
0x0000001c (PA): bl _start (0x94000000)
\end{lstlisting}

\subsection{Data section}

Data sections starts at $VA = 0x40180000$ ($base\_virtual\_address + origin = 0x40180000 + 0$).

VA of \texttt{:data} is $0x40180000$ and VA of \texttt{:end} is $0x40180060$:

\begin{lstlisting}
data {
  org 0x0
  label :data
  dword rand(0, 0xffff),
        rand(0, 0xffff),
        rand(0, 0xffff),
        rand(0, 0xffff),
        rand(0, 0xffff),
        rand(0, 0xffff),
        rand(0, 0xffff),
        rand(0, 0xffff),
        rand(0, 0xffff)
  label :end
  space 1
}
\end{lstlisting}

The data is allocated in the simulator�s physical memory in the following way (from MicroTESK debug output):

\begin{lstlisting}
0x40180000 (PA): .org 0x0
0x40180000 (PA): data:
0x40180000 (PA):
  .dword 0x000000000000C64E, 0x0000000000002658,
         0x0000000000004D63, 0x000000000000DCC6,
         0x00000000000028E6, 0x00000000000043F2,
         0x0000000000008916, 0x00000000000005FD,
         0x000000000000C60F
0x40180048 (PA): end:
0x40180048 (PA): .space 1
\end{lstlisting}

\subsection{Main code}

Starting VA is set by the \texttt{org.} directive to $0x2000$ ($base\_virtual\_address + origin = 0 + 0x2000$).
The PA is equal to VA.
The code is allocated in the simulator�s physical memory in the following way (from MicroTESK debug output):

\begin{lstlisting}
0x00002000 (PA): adr x0, data            (0x10000000)
0x00002004 (PA): adr x1, end             (0x10000001)
0x00002008 (PA): mov x2, x0              (0xAA0003E2)
0x0000200c (PA): ldar x6, [x2, #0]       (0xC8DFFC46)
0x00002010 (PA): mov x3, x6              (0xAA0603E3)
0x00002014 (PA): mov x4, x6              (0xAA0603E4)
0x00002018 (PA): cmp x2, x1, LSL #0      (0xEB01005F)
0x0000201c (PA): b.ge exit_for_0000      (0x5400000A)
0x00002020 (PA): ldar x6, [x2, #0]       (0xC8DFFC46)
0x00002024 (PA): cmp x6, x4, LSL #0      (0xEB0400DF)
0x00002028 (PA): b.gt new_max_value_0000 (0x5400000C)
0x0000202c (PA): cmp x6, x3, LSL #0      (0xEB0300DF)
0x00002030 (PA): b.lt new_min_value_0000 (0x5400000B)
0x00002034 (PA): add x2, x2, #8, LSL #0  (0x91002042)
0x00002038 (PA): b for_0000              (0x14000000)
0x0000203c (PA): mov x4, x6              (0xAA0603E4)
0x00002040 (PA): b next_0000             (0x14000000)
0x0 002044 (PA): mov x3, x6              (0xAA0603E3)
0x00002048 (PA): b next_0000             (0x14000000)
\end{lstlisting}
